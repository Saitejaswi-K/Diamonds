\documentclass[12pt]{beamer}
\beamertemplatenavigationsymbolsempty
\usetheme{Madrid}
\usepackage[utf8]{inputenc}
\usepackage{enumitem}
\usepackage{ulem}

\title{\textbf{DIAMONDS}}
\author[Bvrith]{Y.Vijitha 20WH1A0483 ECE-B\\K.Sai Tejaswi 20WH1A1206 IT-A\\ B.Nandini 20WH1A0514 CSE-A\\ T.Harshitha 20WH1A6637 CSE-AI&ML\\ Ch.Abitha 20WH1A0440 ECE-A}
\date{\textbf{21th October, 2021}}


\begin{document}
    \begin{frame}
        \titlepage
    \end{frame}
    \begin{frame}
	\frametitle{Introduction}
        \begin{itemize}
	    \item Diamonds is played by 3 players using a 52 deck of playing cards. The objective is to win diamond card by bidding it with cards of other suits and score the points according to values alloted to the card.
	\end{itemize}
    \end{frame}
    \begin{frame}
	\frametitle{Rules}
	\begin{itemize}
	    \item 1.Diamond cards are shuffled and kept aside.
	    \item 2.Each player pick each suit
	    \item 3.Initially  a Diamond card is displayed to the players and the playing bidding with the highest value card wins that diamond card and points
	    \item 4.Points are awarded as follows:\\
	    Numbers(2-10) - 3\\
	    Royal(King,Queen,Jack) - 10\\
	    Ace - 11\\
	    Points are divided if 2 players bid with same value card
	    \item 5.Player with highest points is the winner  
	    
	\end{itemize}
    \end{frame}
    \begin{frame}
	\frametitle{Approach}
	\begin{itemize}
	    \item K.Sai Tejaswi: Worked on shuffling, deck seperation, card values and latex\\
	    \begin{figure}
\includegraphics[width=10cm]{p1.png}
\end{figure}

	\end{itemize}
    \end{frame}
    \begin{frame}
	\frametitle{Approach}
	\begin{itemize}
	    \item B.Nandhini: worked on classes player, cards, deck, game and allocation of points\\
	    \begin{figure}
\includegraphics[width=10cm]{Screenshot.png}
\end{figure}
	    \end{itemize}
	    \end{frame}
    \begin{frame}
	\frametitle{Approach}
	\begin{itemize}
	    \item T.Harshitha: Worked on latex basic screen for gaming using pygame\\
\begin{figure}
\includegraphics[width=10cm]{Screenshot (229).png}
\end{figure}
	    
	    \end{itemize}
	    \end{frame}
	    \begin{frame}
	\frametitle{Approach}
	\begin{itemize}
	    \item Y.Vijitha: worked on GUI design and basic screen for gaming\\
	    \begin{figure}
\includegraphics[width=10cm]{Screenshot ui.png}
\end{figure}
	    \end{itemize}
	    \end{frame}
	    
    \begin{frame}
        \frametitle{Learnings}
	\begin{itemize}
	    \item We got to know about Pygame, UI applications. We used object oriented programming concept while writing code
	\end{itemize}
    \end{frame}
    \begin{frame}
	\frametitle{Challenges}
        \begin{itemize}
	    \item It has been difficult to understand how graphics could be included in game. But later on
	    understanding GUI made easy.
        \end{itemize}
    \end{frame}
    \begin{frame}
	\frametitle{Statistics}
        \begin{itemize}
        \item K.Sai Tejaswi\\
	     \item Number of Lines of Code:23\\
	     \item Number of Functions:3\\
	     \item T.Harshitha\\
	     \item Number of Lines of Code:16\\
	     \item Number of Functions:3\\
	     \item Y.Vijitha\\
	     \item Number of Lines of Code:19\\
	     \item Number of Functions:2\\
	     \item B.Nandhini\\
	     \item Number of Lines of Code:108\\
	     \item Number of Functions:12\\
	     
        \end{itemize}
    \end{frame}
    \begin{frame}
	\frametitle{Demo/ Screen Shots}
	\begin{figure}
\includegraphics[width=10cm]{ui.png}
\end{figure}
    \end{frame}
    \begin{frame}
	\begin{center}
	      THANK YOU
	\end{center}
    \end{frame}
\end{document}
