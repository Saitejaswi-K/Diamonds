\documentclass[12pt]{beamer}
\beamertemplatenavigationsymbolsempty
\usetheme{Madrid}
\usepackage[utf8]{inputenc}
\usepackage{enumitem}
\usepackage{ulem}
\item{\center{\large{\textbf{BVRIT HYDERABAD COLLEGE OF ENGINEERING FOR WOMEN}}}
\title{\textbf{26TH oct 2021}}
\title{\textbf{DIAMONDS}}
\author[Bvrith]{Y.Vijitha:20WH1A0483 \\K.Sai Tejaswi:20WH1A1206\\ B.Nandini:20WH1A0514\\ T.Harshitha:20WH1A6637\\ Ch.Abitha:20WH1A0440}
\date{\textbf{TEAM 15}}

\begin{document}

\maketitle

\section{Introduction}


\begin{frame}{\LARGE{\textbf{RULES}}}
        \begin{enumerate}[label = \arabic*)]
	    \item Diamond cards are shuffled and kept aside.
        \item Other Suit cards are shuffled and distributed to the players.
        \item A diamond card is displayed to the players, player bidding with highest value card wins that diamond card and points.
        \item Points are awarded to the players as follows:
        \end{enumerate}
        \begin{itemize}
                \item Numbers(2 - 10): 3
                \item Royals(Jack, Queen, King): 10
                \item Ace: 20
                \item If two players bid with the same value card, the points are divided among them.
        \end{itemize}
\end{frame}

\begin{frame}
	\frametitle{APPROACH}
	\begin{enumerate}[label = \arabic*)]
	    \item At first we defined two functions diamondDeck and remainingDeck
        \item By using random.shuffle() built in function both decks will be shuffled 
        \item Then remainingDeck will be distributed equally among three players by slicing the list.
        \item The values and points are assigned to the cards using dictionaries.
        \item For updating the players score based on the diamond card maximum function is used to return the highest player.
        \item Based on the instructions of the game conditional statements are used to divide scores among players.
	\end{enumerate}
\end{frame}
\begin{frame}
    \frametitle{LEARNINGS}
    \begin{itemize}
        \item[$\bullet$] Git version controls
        \item[$\bullet$] Pygame
        \item[$\bullet$] Reffered to different modules from python Documentation
    \end{itemize}
\end{frame}
\begin{frame}
    \frametitle{CHALLENGES}
    \begin{itemize}
         \item[$\bullet$]While integrating gamecode with the UI we faced difficulties.
         \item[$\bullet$] As instructions for individual deck are different,it was hard for comparing the cards.
    \end{itemize}
\end{frame}
\begin{frame}{References with links}
\begin{itemize}
        \item[$\bullet$] \href{https://www.pygame.org/docs/}{Pygame Documentation}
        \item[$\bullet$] \href{https://docs.python.org/3/tutorial/}{Python Documentation}
        
    \end{itemize}
\end{frame}
\begin{frame}{STATISTICS}\includegraphics{Statistics1.png}
\end{frame}
\begin{frame}{Git Repository}
\begin{figure}
\includegraphics[width=10cm]{GitRepo.png}
\end{figure}
\end{frame}
\begin{frame}{PLAYERS BIDDING CARDS WITH DIAMOND CARD}
\begin{figure}
\includegraphics[width=10cm]{Demo.jpeg}
\end{figure}
\end{frame}
\begin{frame}{START SCREEN}
\begin{figure}
\includegraphics[width=10cm]{title screen.png}
\end{figure}
\end{frame}
\begin{frame}{SHOWING PLAYERS CARDS}
\begin{figure}
\includegraphics[width=10cm]{game window.png}
\end{figure}
\end{frame}
\begin{frame}{EXIT SCREEN}
\begin{figure}
\includegraphics[width=10cm]{end screen.png}
\end{figure}
\end{frame}

\begin{frame}
    \center{\LARGE{\textbf{THANK YOU}}}
\end{frame}


\end{document}
