\documentclass[12pt]{beamer}
\beamertemplatenavigationsymbolsempty
\usetheme{Madrid}
\usepackage[utf8]{inputenc}
\usepackage{enumitem}
\usepackage{ulem}

\title{\textbf{DIAMONDS}}
\author[Bvrith]{Y.Vijitha 20WH1A0483 ECE-B\\K.Sai Tejaswi 20WH1A1206 IT-A\\ B.Nandini 20WH1A0514 CSE-A\\ T.Harshitha 20WH1A6637 CSE-AI&ML\\ Ch.Abitha 20WH1A0440 ECE-A}
\date{\textbf{20st October, 2021}}


\begin{document}
    \begin{frame}
        \titlepage
    \end{frame}
    \begin{frame}
	\frametitle{\LARGE{\textbf{Introduction}}}
        \begin{itemize}
    	    \item Diamonds is played by 3 players using a 52 deck of playing cards. The objective is to win diamond card by bidding it with cards of other suits and score the points according to values alloted to the card.
    	\end{itemize}
    \end{frame}
    \begin{frame}
	\frametitle{\LARGE{\textbf{Rules}}}
    	\begin{enumerate}[label = \arabic*)]
    	    \item 1.Diamond cards are shuffled and kept aside.
    	    \item 2.Each player pick each suit
    	    \item 3.Initially  a Diamond card is displayed to the players and the playing bidding with the highest value card wins that diamond card and points
    	    \item 4.Points are awarded as follows:\\
    	    Numbers(2-10) - 3\\
    	    Royal(King,Queen,Jack) - 10\\
    	    Ace - 11\\
    	    Points are divided if 2 players bid with same value card
    	    \item 5.Player with highest points is the winner  
    	    
    	\end{enumerate}
    \end{frame}
    
    \begin{frame}{\LARGE{\textbf{DAY 1}}}
        \begin{enumerate}[label = \arabic*)]
            \item Tejaswi worked on game code(Shuffling, Deck seperation, Card Values) 
            \item Nandini worked on game code(classes player, card, deck, game) and allocation on points
            \item Harshitha worked on LaTeX and basic screen for game using pygame
            \item Vijitha and Abitha worked on GUI Design
        \end{enumerate}
    \end{frame}
    
    \begin{frame}{\LARGE{\textbf{Technical Stack }}}
        \begin{enumerate}[label = \arabic*)]
            \item \large{Python}
            \item \large{Pygame}
            \item \large{LateX}
            \item \large{GitLab}
        \end{enumerate}
    \end{frame}

\end{document}
